\documentclass[12pt,a4paper]{article}
\usepackage{hyperref}

\title{Finding Optimized Machine Learning Model For Recognizing English Handwritten Digit}
\author{Nowfel Mashnoor\\Roll: 1503069\and Amir Faruk\\Roll: 1503075}

\begin{document}
\date{}
\maketitle

\begin{abstract}
This paper is about the comparison between different Machine Learning models(classifiers) trained and tested on MNIST dataset. For declaring a model as best, we only considered low error score. A standard machine learning library written in Python Programming Language is used during this research. 
\end{abstract}

\section{Introduction}

Handwritten Digit Recognition has been very successful in recent years. A lot of research and studies has been done in recent years on it like Devnagari Handwritten Character Recognition\cite{pal2009comparative}. Handwritten digit recognition technique is used in various fields like PDA, bank cheque, handwritten fields in form etc.\cite{plamondon2000online} Using machine learning technique, which which can be briefly defined as enabling computers make successful predictions using past experiences, \cite{bacstanlar2014introduction} handwritten digit recognition system is greatly improved. Handwritten Digit recognition is a supervised learning algorithm problem. There are many classifier in supervised learning like Neural Network, Decision Tree,  Bayesian Network, Support Vector Machine(SVM), Random Forest etc\cite{kotsiantis2007supervised}. A comparison study has been already done where \textit{Base Linear Classifier, Baseline Nearest Neighbor Classifier, Large Fully Connected Multi-Layer Neural Network, Tangent Distance Clasifier(TDC), LeNet 4 With KNN, Optimal Margin Classifier} are compared among.\cite{lecun1995learning} In our research we are going to compare among algo1, algo2, algo3. We will chose the best classifier among them based on their accuracy on testing set. 

\bibliography{ref} 
\bibliographystyle{ieeetr}

\end{document} 